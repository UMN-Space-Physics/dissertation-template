% #############################################################################
% ############################################################## Appendix: Math
% #############################################################################

\chapter{Math}
  \label{app_math}

As you may recall from \cref{ch_data}, there was some data. 

Now let's look at some math. It's based on work by Lysak\cite{lysak_2013}. 

More on that in \cref{sec_maxeqn}

% =============================================================================
% =============================================================================
% =============================================================================

\section{Maxwell's Equations}
  \label{sec_maxeqn}

Start with \amplaw, swapping out the current for $\; \tensor{\sigma} \cdot 
\vec{E} \;$ per Kirchhoff's formulation of \ohmlaw. Finagle a bit to get it
into a form where we can use an integrating factor. 
\begin{align}
  \tensor{\epsilon} \cdot \ddt \vec{E} &= \oomz \curl{B} -
    \tensor{\sigma} \cdot \vec{E}
\end{align}

This can be finagled to:
\begin{align}
  \label{eqn_intfac_0}
  \lr{ \tensor{\Omega} + \tensor{ \mathbb{I} } \ddt } \cdot \vec{E} &=
    \tensor{V}^2 \cdot \vec{F}
\end{align}

Where $\tensor{ \mathbb{I} }$ is the identity tensor, 
${\vec{F} \equiv \curl{B}}$ and in dipole $x$-$y$-$z$ coordinates, 
\begin{align}
  \label{eqn_intfac_1}
  \tensor{V}^2 &\equiv \oomz \tensor{\epsilon}^{-1} = 
    \mmm{\va^2}{0}{0}
        {0}{\va^2}{0}
        {0}{0}{c^2} &
  & \text{and} &
  \tensor{\Omega} &\equiv \tensor{\epsilon}^{-1} \cdot \tensor{\sigma} = 
    \mmm{ \frac{\sp}{\ep} }{ \frac{-\sh}{\ep} }{0}
        { \frac{\sh}{\ep} }{ \frac{\sp}{\ep} }{0}
        {0}{0}{ \frac{\sz}{\ez} } 
\end{align}

Multiplying through by $\exp \lr{ \tensor{\Omega} \, t }$ and applying the
product rule, \cref{eqn_intfac_1} becomes\footnote{Tensor exponentiation is
analogous to scalar exponentiation\cite{hall_2015}:
$\exp\lr{\tensor{T}} \equiv \displaystyle\sum_n \frac{1}{n!} \tensor{T}^n$. }
\begin{align}
  \label{eqn_intfac_2}
  \ddt \Big( \exp \arg{\tensor{\Omega} \, t} \cdot \vec{E} \, \Big) &=
    \exp \lr{\tensor{\Omega} \, t} \cdot \tensor{V}^2 \cdot \vec{F}
\end{align}

\cref{eqn_intfac_2} can then be integrated over a small time step \dt. Note
that in \cref{eqn_intfac_3}, the $\leftarrow$ operator indicates assignment;
values on the left are new, and those on the right are old. 
\begin{align}
  \label{eqn_intfac_3}
  \vec{E} &\leftarrow \exp \arg{ -\tensor{\Omega} \, \dt } \cdot \vec{E} +
    \dt \, \exp \arg{ -\tensor{\Omega} \, \tfrac{\dt}{2} } \cdot
    \tensor{V}^2 \cdot \vec{F}
\end{align}

Now there's something we can plug into the computer! 




