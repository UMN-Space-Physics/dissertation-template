
% Macros are useful for anything that comes up often. They are defined here,
% and can then be used throughout the document. 

% Bright red reminder. 
\newcommand{\todo}[1]{ \textcolor{red}{TODO --- #1} }

% \ensuremath and \xspace are useful, but sometimes frowned upon. There are 
% some corner cases where they don't behave as expected. 

% Make to-do notes show up in red. 
\newcommand{\dft}[1]{\ensuremath{\overset{\sim}{#1}}\xspace}

% Real and Imaginary prefixes. 
\newcommand{\real}{\ensuremath{\mathbb{R}\mathrm{e}}\xspace}
\newcommand{\imag}{\ensuremath{\mathbb{I}\mathrm{m}}\xspace}

% Names which include special characters. 
\newcommand{\Alfven}{Alfv\'en\xspace}
\newcommand{\Ampere}{Amp\`ere\xspace}
\newcommand{\Schrodinger}{Schr\"odinger\xspace}

% Not sure if it's "Ohm's law" or "Ohm's law"? Use a macro and it's easy to
% change later. 
\newcommand{\ohmlaw}{Ohm's law\xspace}
\newcommand{\amplaw}{\Ampere's law\xspace}
\newcommand{\farlaw}{Faraday's law\xspace}
\newcommand{\maxeqs}{Maxwell's equations\xspace}

% Use underline to indicate vectors and double-underline for tensors. The \vec
% function is already defined -- this overwrites it. 
\renewcommand{\vec}[1]{\ensuremath{\underline{#1}}}
\newcommand{\tensor}[1]{\ensuremath{\underline{\underline{#1}}}}

% Differential operator shortcuts. 
\newcommand{\dd}[1]{\ensuremath{ \frac{\partial}{\partial #1} }\xspace}
\newcommand{\ddt}{\dd{t}\xspace}
\newcommand{\curl}[1]{\ensuremath{ \nabla \times \vec{#1} }\xspace}
\renewcommand{\div}[1]{\ensuremath{ \nabla \cdot \vec{#1} }\xspace}
\newcommand{\grad}[1]{\ensuremath{ \nabla #1 }\xspace}

% Properly-sized matching left and right parentheses. 
\newcommand{\lr}[1]{ \left( #1 \right) }

% Squish "\delta t" together a bit to make it look better. 
\newcommand{\dt}{\ensuremath{\delta \hspace{-0.1em} t}\xspace}

% For the \SI command, make eV look better by shifting it slightly left. 
\DeclareSIUnit\electronvolt{e\hspace{-0.08em}V}

% Plasma frequency, Alfven speed, and other terms that show up a lot in
% dispersion relations. 
\newcommand{\op}{\ensuremath{\omega_P}\xspace}
\newcommand{\va}{\ensuremath{v_A}\xspace}
\newcommand{\ep}{\ensuremath{\epsilon_\bot}\xspace}
\newcommand{\ez}{\ensuremath{\epsilon_0}\xspace}
\newcommand{\mz}{\ensuremath{\mu_0}\xspace}
\newcommand{\oomz}{\ensuremath{ \frac{1}{\mz} }\xspace}
\newcommand{\sz}{\ensuremath{\sigma_0}\xspace}
\newcommand{\sh}{\ensuremath{\sigma_H}\xspace}
\renewcommand{\sp}{\ensuremath{\sigma_P}\xspace}

% Shorthand for a 2x2 matrix. Used as (whitespace/newlines optional):
% \mm{\cos\theta}{\-sin\theta}
%    {\sin\theta}{\cos\theta}
\newcommand{\mm}[4]{ \left[ \begin{array}{cc}
    #1 & #2 \\
    #3 & #4
  \end{array} \right] }

% 3x3 matrix. 
\newcommand{\mmm}[9]{ \left[ \begin{array}{ccc}
    #1 & #2 & #3 \\
    #4 & #5 & #6 \\
    #7 & #8 & #9
  \end{array} \right] }

% 2x1 matrix, or 2-vector. 
\newcommand{\vv}[2]{ \left[ \begin{array}{c}
    #1 \\
    #2
  \end{array} \right] }



